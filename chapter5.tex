\chapter{Phương trình vi phân I-tô}
\section{Phương trình vi phân ngẫu nhiên}
\begin{defn}[Phương trình vi phân ngẫu nhiên]
    Cho $X = (X_t)_{t \in [0, T]}$ là một quá trình ngẫu nhiên. Khi đó phương trình 
    \begin{align}\label{defn ptvpnn}
        dX_t &= a(t,X_t) dt + \sigma(t,X_t)dB_t, \quad t\in [0,T]\\
        X_0  &= x \in \R. \nonumber
    \end{align}
\end{defn}
\begin{defn}
    Quá trình ngẫu nhiên $X = (X_t)_{t\in [0,T]}$ được gọi là nghiệm của phương trình vi phân ngẫu nhiên \ref{defn ptvpnn} nếu 
    \begin{enumerate}
        \item $X$ là $\ff-$tương thích, tức $X_t$ là $\ff_t-$đo được.
        \item Các tích phân $\displaystyle \int_0^t{a(s,X_s)ds}$ và $\displaystyle \int_0^t{\sigma(s,X_s)dB_s}$ tồn tại và thoả mãn 
        \[X_t = x + \int_0^t{a(s,X_s)ds} + \int_0^t{\sigma(s,X_s)dB_s}\quad \forall t \in [0,T].\]
    \end{enumerate}
\end{defn}
\section{Phương trình tuyến tính tổng quát}
\begin{defn}[Phương trình tuyến tính tổng quát]
    Phương trình tuyến tính tổng quát có dạng
    \begin{align*}
        dX_t &= (a(t)+b(t)X_t)dt + (c(t)+e(t)X_t)dB_t\\
        X_0&= x \in \R.
    \end{align*}
\end{defn}
\begin{sol*}
    \begin{enumerate}
        \item Xét phương trình 
        \begin{align*}
            dY_t &= b(t)Y_tdt + e(t)Y_tdB_t \quad \forall t \in [0,T]\\
            Y_0 &= 1.
        \end{align*}
        Áp dụng công thức vi phân I-tô cho $Z_t = \ln{Y_t}$ ta được
        \begin{align*}
            dZ_t &= \dfrac{1}{Y_t}dY_t + \dfrac{1}{2}\dfrac{-1}{Y_t^2}(dY_t)^2\\
            &= \dfrac{b(t)Y_tdt + e(t)Y_tdB_t}{Y_t} + \dfrac{-1}{2Y_t^2}(b(t)Y_tdt + e(t)Y_tdB_t)^2\\
            &= b(t)dt + e(t)dB_t - \dfrac{e^2(t)}{2}dt\\
            &= \left(b(t)- \dfrac{e^2(t)}{2}\right)dt + e(t)dB_t.
        \end{align*}
        Từ đó thu được
        \begin{align*}
            Z_t &= Z_0 + \int_0^t{\left(b(s)- \dfrac{e^2(s)}{2}\right)ds} + \int_0^t{e(s)dB_s}  \\
            Z_0 &= \ln{Y_0} = \ln{1} = 0.
        \end{align*}
        Do đó \[Y_t = \exp{Z_t} = \exp\left(\int_0^t{\left(b(s)- \dfrac{e^2(s)}{2}\right)ds} + \int_0^t{e(s)dB_s}\right).\]
        \item Đặt $U_t = \dfrac{X_t}{Y_t}$, áp dụng công thức vi phân I-tô ta được
        \begin{align*}
            dU_t &= \dfrac{1}{Y_t}dX_t + \dfrac{-X_t}{Y_t^2}dY_t + \dfrac{1}{2}\left(0\cdot (dX_t)^2 + 2\cdot \dfrac{-1}{Y_t^2}dX_tdY_t + \dfrac{2X_t}{Y_t^3}(dY_t)^2\right).
        \end{align*}
        Trong đó $dX_tdY_t = e(t)Y_t[c(t)+e(t)X_t]dt$ và $(dY_t)^2 = e^2(t)Y_t^2dt$, nên
        \begin{align*}
            dU_t &= \left(\dfrac{a(t)-c(t)e(t)}{Y_t}\right)dt + \dfrac{c(t)}{Y_t}dB_t\\
            U_t  &= U_0 + \int_0^t \left(\dfrac{a(s)-c(s)e(s)}{Y_s}\right)ds + \int_0^t\dfrac{c(s)}{Y_s}dB_s\\
            U_0 &= \dfrac{X_0}{Y_0} = x.
        \end{align*}
        Từ đó ta thu được $X_t = U_tY_t$.
    \end{enumerate}
\end{sol*}
\begin{exam*}
    Giải các phương trình sau
    \begin{enumerate}
        \item 
        $\begin{cases}
            dX_t &= -X_tdt + dB_t, \quad t\in [0,T] \\
            X_0 &=  x
         \end{cases}$
        \item 
        $\begin{cases}
            dX_t &= 2dt + X_tdB_t, \quad t\in [0,T] \\
            X_0 &=  x
        \end{cases}$
    \end{enumerate}
\end{exam*}
\begin{sol*}
\begin{enumerate}
    \item Xét phương trình: 
    $\begin{cases}
            dY_t &= -Y_tdt \\
            Y_0 &=  1
    \end{cases}$\\
    Đặt: $Z_t = lnY_t$. Áp dụng công thức vi phân Itô ta có
    \[dZ_t = \dfrac{1}{Y_t}dY_t -\dfrac{1}{2Y_t^2}(dY_t)^2 = -dt.\] Suy ra \[Z_t = Z_0 + \displaystyle\int\limits_0^t-sds = -t.\] Do đó $Y_t = e^{-t}.$\\
    Đặt: $U_t = \dfrac{X_t}{Y_t}$ thì $\begin{cases}
            dU_t &= \Big(\dfrac{0}{Y_t} - \dfrac{1\cdot 0}{Y_t}\Big)dt + \dfrac{1}{Y_t}dB_t = e^tdB_t \\
            U_0 &=  \dfrac{X_0}{Y_0} = x
    \end{cases}$\\
    Suy ra $U_t = x + \displaystyle\int\limits_0^te^sdB_s$\\
    Vậy $X_t = Y_t \cdot U_t = e^{-t}\bigg(x+\displaystyle\int\limits_0^te^sdB_s\bigg) = xe^{-t} + \displaystyle\int\limits_0^t e^{-(t-s)}dB_s.$\\
    \remarkname: \begin{align*}
        \E X_t &= xe^{-t}\\
        \Var(X_t) &= E(X_t - EX_t)^2\\
        &= \E\bigg(\displaystyle\int\limits_0^t e^{-(t-s)}dB_s\bigg)^2\\
        &= \E\bigg(\displaystyle\int\limits_0^t e^{-2(t-s)}ds\bigg)\\
        &= \displaystyle\int\limits_0^t e^{-2(t-s)}ds\\
        &= \dfrac{1-e^{-2t}}{2}.
    \end{align*}
    \item Xét phương trình
    $\begin{cases}
            dY_t &= Y_tdB_t \\
            Y_0 &=  1
    \end{cases}$\\
    Đặt: $Z_t = \ln Y_t$. Áp dụng công thức vi phân Itô, ta có
    \begin{align*}
        dZ_t &= \dfrac{1}{Y_t}dY_t - \dfrac{1}{2Y_t^2}(dY_t)^2\\
        &= dB_t - \dfrac{1}{2Y_t^2}Y_t^2dt\\
        &= dB_t\\
        &=\dfrac{1}{2}dt.
    \end{align*}
    Do đó
    \begin{align*}
        Z_t &= Z_0 + \displaystyle\int\limits_0^tdB_s -\dfrac{1}{2}\displaystyle\int\limits_0^t ds\\
        &= -\dfrac{1}{2}t + \displaystyle\int\limits_0^tdB_s\\
        &= -\dfrac{1}{2}t + B_t.
    \end{align*}
    Nên \[Y_t = e^{-\dfrac{1}{2}t + B_t}.\]
    Xét $U_t = \dfrac{X_t}{Y_t.}$ Áp dụng công thức Itô, ta có: \[dU_t = \bigg(\dfrac{2}{Y_t} - \dfrac{0}{Y_t}\bigg)dt + \dfrac{0}{Y_t}dB_t = \dfrac{2}{Y_t}dt\]
    Do đó 
    \begin{align*}
        U_t &= U_0 + 2\displaystyle\int\limits_0^t\dfrac{1}{Y_s}ds\\
        &= x + 2\displaystyle\int\limits_0^t\dfrac{1}{Y_s}ds\\
        &= x + 2\displaystyle\int\limits_0^t e^{\dfrac{s}{2} - B_s}ds.
    \end{align*}
    Vậy $X_t = Y_t \cdot U_t = e^{-\dfrac{1}{2}t + B_t}\bigg(x + 2\displaystyle\int\limits_0^t e^{\dfrac{s}{2} - B_s}ds\bigg).$
\end{enumerate}    
\end{sol*}
\section{Phương trình bán tuyến tính}
\begin{defn}[Phương trình bán tuyến tính] Phương trình bán tuyến tính có dạng    
        \[\begin{cases}
            dX_t &= b(t,X_t)dt+ \sigma(t) X_t dB_t, t \in [0,T]\\
            X_0 &= x \in \R
         \end{cases}\]     
\end{defn} 
\begin{sol*} 
    \begin{enumerate}
        \item Xét \qtnn:
        \[Y_t = \exp \left(-\int\limits_0^t \sigma(s) dB_s + \dfrac{1}{2} \int\limits_0^t \sigma^2(s) ds \right). \]
        Áp dụng công thức vi phân I-tô ta có:
        \begin{align*}
            dY_t &= Y_t \left(-\sigma(t) dB_t + \dfrac{1}{2} \sigma^2(t) dt \right) + \dfrac{1}{2} Y_t \sigma^2(t) dt \\
            &= \sigma^2(t) Y_t dt - \sigma(t) Y_t dB_t.           
        \end{align*}
        \item Xét $X_t = X_t \cdot Y_t$. Áp dụng công thức vi phân Itô ta có:
        \begin{align*}
            dZ_t &= Y_t dX_t + X_t dY_t + dX_t dY_t\\
            &= Y_t \left(b(t,X_t)dt + \sigma(t) X_t dB_t \right) + X_t \left( \sigma^2(t) Y_t dt - \sigma(t) Y_t dB_t  \right) - \sigma^2(t) X_t Y_t dt  \\
            &= Y_t b(t,X_t) dy.
        \end{align*}
        Như vậy, $Z_t$ là nghiệm của phương trình vi phân tất định
        \[dZ_t = Y_t b\left(t, \dfrac{Z_t}{Y_t} \right) dt.\]
        Tuỳ vào hàm $b(t,X_t)$ ta có thể tìm $Z_t$ tường minh. Khi đó kết luận nghiệm là $X_t = \dfrac{Z_t}{Y_t}$.
    \end{enumerate}
\end{sol*}
\examplename. Giải phương trình: 
\[\begin{cases}
    dX_t &= \dfrac{dt}{X_t} + X_t dB_t, t \in [0,t]\\
    X_0 &= x >0.
\end{cases}\]
\begin{sol*}
    Xét \qtnn: \[Y_t = \exp \left(-\int\limits_0^t 1 dB_s + \dfrac{1}{2} \int\limits_0^t 1^2 ds \right) = \e^{\dfrac{t}{2} - B_t}. \]
    Coi $Y_t = f(t,B_t) = \e^{\dfrac{t}{2} - B_t}$. Áp dụng công thức vi phân Itô ta có:
    \begin{align*}
        dY_t &= \dfrac{1}{2} \e^{\dfrac{t}{2} - B_t} dt - \e^{\dfrac{t}{2} - B_t} dB_t + \dfrac{1}{2} \e^{\dfrac{t}{2} - B_t} dt \\
        &= \e^{\dfrac{t}{2} - B_t} dt - \e^{\dfrac{t}{2} - B_t} dB_t \\
        &= Y_t dt - Y_t dB_t.
    \end{align*}
    Xét $Z_t = X_t \cdot Y_t$. Áp dụng công thức vi phân Itô ta có:
    \begin{align*}
        dZ_t &= Y_t \left(\dfrac{dt}{X_t} + X_t dB_t \right) + X_t (Y_t dt - Y_t dB_t) - X_t Y_t dt\\
        &= \dfrac{Y_t}{X_t} dt.
    \end{align*}
    Thay $X_t = \dfrac{Z_t}{Y_t}$ vào biểu thức trên ta được:
    \begin{align*}
         dZ_t & = \dfrac{Y_t^2}{Z_t} dt \\
         Z_t dZ_t &= Y_t^2 dt \\
         \int\limits_0^t Z_s dZ_s &= \int\limits_0^t Y_s^2 ds \\        
         \dfrac{Z_s^2}{2} \big|_0^t &= \int\limits_0^t \e^{s-2B_s}ds.
    \end{align*}
    Do đó:
    \begin{align*}
        Z_t^2 &= Z_0^2 + 2\int\limits_0^t \e^{s-2B_s}ds\\
        &= x^2 + 2\int\limits_0^t \e^{s-2B_s}ds.
    \end{align*}
    Vì $X_0 = x>0$ nên \[Z_t = \sqrt{x^2 + 2\int\limits_0^t \e^{s-2B_s}ds}.\]
    Vậy nghiệm cần tìm là:
    \[X_t = \dfrac{Z_t}{Y_t} = \dfrac{\sqrt{x^2 + 2\int\limits_0^t \e^{s-2B_s}ds}}{\e^{\dfrac{t}{2} - B_t}}.\]
\end{sol*}
\examplename: Giải các phương trình sau:
\begin{enumerate}
    \item $\begin{cases}
        dX_t &= X_t^\alpha dt + \sigma X_t dB_T, t \in [0,T] \\
        X_0 &= x > 0
    \end{cases}$, với $\alpha, \sigma$ là các hằng số thực.
    \item $\begin{cases}
        dX_t &= K(\alpha - \ln X_t)X_t dt + \sigma X_t dB_T, t \in [0,T] \\
        X_0 &= x > 0
    \end{cases}$, với $K, \alpha, \sigma > 0$.
\end{enumerate}
\section{Phương trình vi phân tất định}
Xét phương trình vi phân tất định
\begin{align*}
    dx_t &= x_t(b+ax_t)dt\quad a,b \in \R,\\
    x_0 &= x >0,
\end{align*}
có nghiệm 
\[ x_t = \dfrac{b}{-a+e^{-bt}\dfrac{b+ax}{x}}\cdot\]
Nếu $a<0<b$ thì $-a+e^{-bt}\dfrac{b+ax}{x} = a(e^{-bt}-1)+\dfrac{be^{-bt}}{x} >0$, khi đó $x_t$ luôn tồn tại với mọi $t \geq 0$. Nếu $a,b>0$ thì $x_t$ chỉ tồn tại từ $t = 0$ cho đến thời điểm $T = \dfrac{-1}{b}\ln{\dfrac{ax}{ax+b}}$.

Bằng cách thêm một yếu tố ngẫu nhiên, ta xét phương trình vi phân ngẫu nhiên sau
\begin{align*}
    dX_t &= X_t(b+aX_t)dt + \varepsilon X_t^2dB_t\\
    X_0 &= x >0\\
    a,b,\varepsilon & \in \R.
\end{align*}
Ta sẽ chỉ ra phương trình trên luôn có nghiệm hữu hạn với mọi $t \geq 0$.
\subsection{Sự tồn tại và tính duy nhất nghiệm}
\begin{lem}[Gronwall]
    Giả sử hàm $u: [a,b] \to \R$ là một hàm không âm thoả mãn 
    \[u(t) \leq \varepsilon + m\int_0^tu(s)ds\quad \forall t \in [a,b]\]
    với $\varepsilon, m >0$. Khi đó ta có ước lượng sau
    \[u(t) \leq \varepsilon e^{t-a} \quad \forall t\in [a,b].\]
\end{lem}
\begin{thm}
    Nếu phương trình vi phân ngẫu nhiên
    \begin{align*}
        dX_t &= b(t,X_t)dt + \sigma(t,X_t)dB_t,~t \in [0,T]\\
        X_0 &= x \in \R
    \end{align*}
    Với $b, \sigma: [0,T]\times \R \to \R$ thoả mãn tồn tại $L  > 0$ sao cho
    \begin{enumerate}
        \item (Tính chất Lipchitz)
        \[|b(t,x)-b(t,y)|+ |\sigma(t,x)-\sigma(t,y)| \leq L|x-y|~\forall t\in [0,T],~x,y\in \R.\]
        \item (Tính chất tăng trưởng tuyến tính)
        \[|b(t,x)| + |\sigma(t,x)| \leq L(1+|x|) ~\forall t \in [0,T],x\in \R.\]
    \end{enumerate}
    thì nó luôn có nghiệm duy nhất.
\end{thm}
\begin{proof}
\begin{enumerate}
    \item (Tính duy nhất nghiệm)\\
    Giả sử $(X_t)_{t \in [0,T]}$ và $(Y_t)_{t \in [0,T]}$ là hai nghiệm của phương trình trên. Khi đó với mọi $t \in [0,T]$ thì
    \begin{align*}
        X_t &= x + \int_0^t{b(s,X_s)ds} + \int_0^t{\sigma(s,X_s)dB_s}\\
        Y_t &= x + \int_0^t{b(s,Y_s)ds} + \int_0^t{\sigma(s,Y_s)dB_s}.
    \end{align*}
    Do đó 
    \[X_t-Y_t = \int_0^t{[b(s,X_s)-b(s,Y_s)]ds} + \int_0^t{[\sigma(s,X_s)-\sigma(s,Y_s)]dB_s}.\]
    Suy ra 
    \[|X_t-Y_t| \leq \int_0^t{\left|b(s,X_s)-b(s,Y_s)\right|ds} + \left|\int_0^t{[\sigma(s,X_s)-\sigma(s,Y_s)]dB_s}\right|.\]
    Dẫn đến 
    \[|X_t-Y_t|^2 \leq 2\left[\int_0^t{\left|b(s,X_s)-b(s,Y_s)\right|ds}\right]^2 + 2\left[\int_0^t{[\sigma(s,X_s)-\sigma(s,Y_s)]dB_s}\right]^2.\]
    Mà theo bất đẳng thức Holder ta có
    \[\int_0^t{\left|b(s,X_s)-b(s,Y_s)\right|ds} \leq \left(\int_0^t{\left|b(s,X_s)-b(s,Y_s)\right|^2ds}\right)^{1/2} \left(\int_0^t{1^2ds}\right)^{1/2}\]
    Suy ra 
    \begin{align*}
        \left[\int_0^t{\left|b(s,X_s)-b(s,Y_s)\right|ds}\right]^2 
        &\leq t\int_0^t{\left|b(s,X_s)-b(s,Y_s)\right|^2ds}\\
        &\leq t\int_0^t{L^2\left|X_s-Y_s\right|^2ds}
    \end{align*}
    Còn theo công thức đẳng cự I-tô thì
    \begin{align*}
        \E\left[\int_0^t{[\sigma(s,X_s)-\sigma(s,Y_s)]dB_s}\right]^2 
        &= \E\int_0^t{[\sigma(s,X_s)-\sigma(s,Y_s)]^2ds}\\
        & \leq \E\int_0^t{L^2|X_s-Y_s|^2ds}.
    \end{align*}
    Từ đó
    \begin{align*}
        \E|X_t-Y_t|^2 &\leq 2L^2(t+1)\E\int_0^t{|X_s-Y_s|^2ds}\\
        & \leq 2L^2(T+1)\int_0^t{\E|X_s-Y_s|^2ds}.
    \end{align*}
    Nếu đặt $u(t) = \E|X_t-Y_t|^2$ thì
    \[u(t) \leq 2L^2(T+1)\int_0^t{u(s)ds}~\forall t \in [0,T].\]
    Từ đó áp dụng bổ đề Gronwall ta được
    \[0 \leq u(t) \leq 0\cdot \exp\left(2L^2(T+1)(t-0)\right),\]
    chứng tỏ $u(t) =\E|X_t-Y_t|^2 = 0$, tức $X_t \overset{hcc}{=} Y_t$ với mọi $t \in [0,T]$.
    \item (Sự tồn tại nghiệm)
    Xét dãy xấp xỉ Picard 
    \begin{align*}
        X_t^{(0)} &= x ~\forall t \in [0,T]\\
        X_t^{(n+1)} &= x + \int_0^t{b(s,X_s^{(n)})ds} + \int_0^t{\sigma(s,X_s^{(n)})dB_s}.
    \end{align*}
    Ta có
    \begin{align*}
        X_t^{(n+1)} - X_t^{(n)} 
        &= \int_0^t{[b(s,X_s^{(n)})-b(s,X_s^{(n-1)})]ds} \\
        &+ \int_0^t{[\sigma(s,X_s^{(n)})-\sigma(s,X_s^{(n-1)})]dB_s}
    \end{align*}
    Lập luận tương tự như phần trước, ta có ước lượng sau
    \begin{align*}
        \E\left|X_t^{(n+1)}-X_t^{(n)}\right|^2 
        &\leq \underbrace{2L^2(T+1)}_{M}\int_0^t{\E\left|X_s^{(n)}-X_s^{(n-1)}\right|^2ds}.
    \end{align*}
    Trong đó
    \begin{align*}
        X_t^{(1)}-X_t^{(0)}
        &= \int_0^t{b(s,X_s^{(0)}ds} +\int_0^t{\sigma(s,X_s^{(0)})dB_s}\\
        &= \int_0^t{b(s,x)ds} +\int_0^t{\sigma(s,x)dB_s}\\
    \end{align*}
    Nên 
    \begin{align*}
        \E\left|X_t^{(1)}-X_t^{(0)}\right|^2 
        &\leq 2\left(\int_0^t{b(s,x)ds}\right)^2+2\left(\E\int_0^t{\sigma(s,x)dB_s}\right)^2\\
        &\leq 2t\int_0^t|b(s,x)|^2ds+2\E\int_0^t{\sigma^2(s,x)ds}\\
        &\leq 2T\int_0^t{L^2(1+|x|)^2ds} + 2\int_0^t{L^2(1+|x|)^2ds}\\
        &\leq 2TL^2(1+|x|)^2t + 2L^2(1+|x|)^2t\\
        &\leq \underbrace{2L^2(1+|x|)^2T(T+1)}_{C}.
    \end{align*}
    Tương tự thì 
    \begin{align*}
        \E\left|X_t^{(2)}-X_t^{(1)}\right|^2 
        &\leq M\int_0^t{\E\left|X_s^{(1)}-X_s^{(0)}\right|^2ds}\\
        &\leq M\int_0^t{Cds} = MCt.\\
        \E\left|X_t^{(3)}-X_t^{(2)}\right|^2 
        &\leq M\int_0^t{\E\left|X_s^{(2)}-X_s^{(1)}\right|^2ds}\\
        &\leq M\int_0^t{MCsds} = M^2C\dfrac{t^2}{2}.\\
        \E\left|X_t^{(4)}-X_t^{(3)}\right|^2 
        &\leq M\int_0^t{\E\left|X_s^{(3)}-X_s^{(2)}\right|^2ds}\\
        &\leq M\int_0^t{M^2C\dfrac{s^2}{2}ds} = M^3C\dfrac{t^3}{3!}.\\
        &\cdots\\
        \E\left|X_t^{(n+1)}-X_t^{(n)}\right|^2 
        &\leq M^nC\dfrac{t^n}{n!} ~\forall t \in [0,T].
    \end{align*}
    Từ đó với mọi $m>n$, ta có
    \begin{align*}
        \left|X_t^{(m)}-X_t^{(n)}\right|^2
        &\leq \left(\left|X_t^{(m)}-X_t^{(m-1)}\right| + \ldots + \left|X_t^{(n+1)}-X_t^{(n)}\right|\right)^2\\
        &\leq \left(\left|X_t^{(m)}-X_t^{(m-1)}\right|^2  + \ldots + \left|X_t^{(n+1)}-X_t^{(n)}\right|^2\right) \cdot \underbrace{(1^2+\cdots+1^2)}_{m-n}.
    \end{align*}
    Vì vậy 
    \begin{align*}
        \E\left|X_t^{(m)}-X_t^{(n)}\right|^2
        &\leq (m-n)\left(\E\left|X_t^{(m)}-X_t^{(m-1)}\right|^2  + \ldots + \E\left|X_t^{(n+1)}-X_t^{(n)}\right|^2\right) \\
        &\leq (m-n) \left(M^{m-1}C\dfrac{t^{m-1}}{(m-1)!}+\ldots+M^nC\dfrac{t^n}{n!}\right)\\
        &\leq (m-n)C \left(\dfrac{(MT)^{m-1}}{(m-1)!}+\ldots+\dfrac{(MT)^n}{n!}\right) \longrightarrow 0 \text{ khi } m,n \to \infty.
    \end{align*}
    Dẫn đến dãy Picard ở trên hội tụ, lấy giới hạn khi $n \to \infty$ cho biểu thức truy hồi dãy trên ta suy ra phương trình đã cho có nghiệm.
    Tiếp theo ta ước lượng moment nghiệm bằng cách sử dụng định lý sau
    \begin{thm}
        Giả sử các hệ số $b$ và $\sigma$ là Lipschitz và tăng trưởng tuyến tính thì 
        \[\E|X_t|^p < \infty ~\forall t \in [0,T], p \geq 2.\]
    \end{thm}
    Vì với mọi $a,b,c$ và $p>1$ thì \[(|a|+|b|+|c|)^p \leq 3^{p-1}(|a|^p+|b|^p+|c|^p).\]
    Nên với \[X_t = x+\int_0^t{b(s,X_s)ds}+\int_0^t{\sigma(s,X_s)dB_s}\]
    thì 
    \begin{align*}
        |X_t|^p &\leq \left(|x|+\left|\int_0^t{b(s,X_s)ds}\right|+\left|\int_0^t{\sigma(s,X_s)dB_s}\right|\right)^p\\
        &\leq 3^{p-1}\left(|x|^p+\left|\int_0^t{b(s,X_s)ds}\right|^p+\left|\int_0^t{\sigma(s,X_s)dB_s}\right|^p\right)\\
    \end{align*}
    Do đó 
    \begin{align*}
        \E|X_t|^p
        &\leq 3^{p-1}\left(\E|x|^p+\E\left|\int_0^t{b(s,X_s)ds}\right|^p+\E\left|\int_0^t{\sigma(s,X_s)dB_s}\right|^p\right)
    \end{align*}
    Mà theo bất đẳng thức Holder thì
    \begin{align*}
        \left|\int_0^t{b(s,X_s)ds}\right| &\leq \left(\int_0^t{\left|b(s,X_s)\right|^pds}\right)^{1/p} \left(\int_0^t{1^qds}\right)^{1/q}
    \end{align*}
    nên 
    \begin{align*}
        \left|\int_0^t{b(s,X_s)ds}\right|^p 
        &\leq \left(\int_0^t{\left|b(s,X_s)\right|^pds}\right)t^{p-1}\\
        &\leq \left(\int_0^t{\left|b(s,X_s)\right|^pds}\right)T^{p-1}\\
        &\leq \left(\int_0^t{\left[L(1+|X_s|)\right]^pds}\right)T^{p-1}.
    \end{align*}
    Mặt khác theo bất đẳng thức $B-D-G$ và Holder thì
    \begin{align*}
        \E\left|\int_0^t{\sigma(s,X_s)dB_s}\right|^p
        &\leq \E\left|\int_0^t{|\sigma(s,X_s)|^2ds}\right|^{p/2}\\
        &\leq T^{P/2-1}\int_0^t{\E|\sigma(s,X_s)|^pds}
    \end{align*}
    Từ đó suy ra 
    \begin{align*}
        \E|X_t|^p 
        &\leq 3^{p-1}\left(|x|^p + T^{p-1}L^p\int_0^t{\E(1+|X_s|)^pds}  + T^{p/2-1}L^p\int_0^t{\E(1+|X_s|)^pds}\right)\\
        &\leq 3^{p-1}\left(|x|^p + (T^{p-1}+T^{p/2-1})L^p\int_0^t{\E(1+|X_s|)^pds}\right)\\
        &\leq 3^{p-1}\left(|x|^p + (T^{p-1}+T^{p/2-1})L^p\int_0^t{\E[2^{p-1}(1+|X_s|^p)]ds}\right)\\
        &\leq 3^{p-1}\left(|x|^p + (T^{p-1}+T^{p/2-1})L^p2^{p-1}\int_0^t{(1+\E|X_s|^p)ds}\right)\\
        &\leq 3^{p-1}\left(|x|^p + (T^{p-1}+T^{p/2-1})L^p2^{p-1}\left[T+\int_0^t{\E|X_s|^pds}\right]\right)\\
        &\leq \varepsilon + m \int_0^t{\E|X_s|^pds}\quad \text{với } \varepsilon, m \text{ nào đó}.
    \end{align*}
    Sử dụng bổ đề Gronwall ta được
        \[\E|X_t|^p \leq \varepsilon e^{mt}~\forall t \in [0,T].\]
    Vì vậy $\E|X_t|^p < \infty~\forall t \in [0,T].$
\end{enumerate}
\end{proof}








